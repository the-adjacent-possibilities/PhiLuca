\documentclass[11pt,a4paper]{article}
\usepackage[utf8]{inputenc}
\usepackage[T1]{fontenc}
\usepackage{lmodern}
\usepackage{amsmath,amssymb,amsthm}
\usepackage{mathtools}
\usepackage{physics}
\usepackage{geometry}
\usepackage{hyperref}
\usepackage{microtype}
\usepackage{parskip}
\usepackage{titlesec}
\usepackage{authblk}
\usepackage{csquotes}

\geometry{margin=1in}

\hypersetup{
    colorlinks=true,
    linkcolor=blue,
    citecolor=blue,
    urlcolor=blue,
    pdftitle={Theoria Omnia (ESQET): Emergent Spacetime Quantum-Entanglement Theory},
    pdfauthor={Marco A. Rocha Jr.}
}

\title{\vspace{-2em}\LARGE\textbf{Theoria Omnia (ESQET)}\\
\large Emergent Spacetime Quantum-Entanglement Theory\\
A Proposed Unification of General Relativity and Quantum Mechanics\\
Through the Dynamics of an Information-Theoretic Scalar Field}

\author{Marco A. Rocha Jr.}
\affil{\small Independent Researcher}
\date{Version: Hybrid -- No rounding of constants or derivations\\ December 2025}

\titleformat{\section}{\Large\bfseries\centering}{}{0em}{}
\titleformat{\subsection}{\large\bfseries}{}{0em}{}

\begin{document}

\begin{abstract}
\noindent\textbf{Abstract}\\
This paper introduces \textbf{Emergent Spacetime Quantum-Entanglement Theory (ESQET)}, a unification framework proposing that spacetime, gravity, and coherent quantum behavior arise from dynamics in a fundamental, dimensionless \textbf{Spacetime Information Field} denoted \( \mathcal{S} \). Unlike approaches that quantize gravity or geometrize quantum mechanics, ESQET treats both as emergent phenomena generated by a single underlying information-theoretic substrate.

At the heart of ESQET is the \textbf{Quantum Coherence Function} \( \mathcal{F}_{QC} \), a nonlocal operator that regulates the degree to which spatial regions share correlated informational states. The constants \( \pi \) and \( \varphi \) (the golden ratio) appear not as mathematical coincidences but as structural invariants governing stable coherence gradients, giving rise to what this paper terms \textbf{Golden Gravity}.

ESQET yields four falsifiable predictions involving entanglement-dependent curvature effects, scale-dependent deviations in gravitational acceleration, novel polarization signatures in cosmological background radiation, and a proposed extension to LIGO enabling detection of coherence-mode gravitational signatures. The theory is speculative yet mathematically disciplined and designed specifically to be testable.
\end{abstract}

\section{Introduction}
Physics has long been divided between two incompatible frameworks:
\begin{itemize}
  \item \textbf{General Relativity (GR)} describes gravity as curvature in smooth spacetime.
  \item \textbf{Quantum Mechanics (QM)} describes microscopic reality as discrete, probabilistic, and fundamentally nonlocal.
\end{itemize}

Attempts to quantize gravity (string theory, loop quantum gravity, etc.) have not yielded empirical predictions. ESQET reverses the usual assumption:

\begin{quote}
\textbf{Gravity is not fundamental.} Gravity is the macroscopic curvature signature of coherence dynamics in a deeper information-theoretic field \( \mathcal{S} \).
\end{quote}

Spacetime geometry emerges through self-consistent configurations of \( \mathcal{F}_{QC} \), analogous to how temperature emerges from molecular statistics.

\section{The Spacetime Information Field (\( \mathcal{S} \))}
We define \( \mathcal{S}(x,y,z,t) \) as a \textbf{dimensionless scalar} that encodes the informational connectivity of all points in spacetime. \( \mathcal{S} \) does \emph{not} reside in spacetime; spacetime is a smooth manifold \emph{defined by gradients in \( \mathcal{S} \)}.

\begin{axiom}[Nonlocality]
\( \mathcal{S} \) has instantaneous global coherence (non-signaling), meaning \( \mathcal{S} \) can adjust globally without violating relativistic locality because it carries no energy and no momentum.
\end{axiom}

\begin{axiom}[Gradient Stability]
Stable regions of spacetime arise where
\[
\nabla \mathcal{S} \to 0.
\]

\begin{axiom}[Energy–Curvature Encoding]
Local curvature arises from how energy-momentum tensors perturb \( \mathcal{S} \), not the reverse:
\[
G_{\mu\nu} = \alpha\,\partial_{\mu}\partial_{\nu}\mathcal{S} + \beta\,(\partial_{\mu}\mathcal{S})(\partial_{\nu}\mathcal{S})
\]
with \( \alpha,\beta \) dimension-preserving constants (no rounding).
\end{axiom}

This is the first bridge between information field dynamics and geometry.

\section{The Quantum Coherence Function (\( \mathcal{F}_{QC} \))}
The key operator is
\[
\mathcal{F}_{QC} = \oint_{\gamma} e^{i\pi\,\Delta\mathcal{S}}\,\varphi^{-\Delta\mathcal{S}}\,d\gamma.
\]
\begin{itemize}
  \item \( \pi \) enforces phase-cycle closure → stabilizes oscillatory quantum states.
  \item \( \varphi \) (golden ratio) enforces minimum-energy coherent packing across scales.
\end{itemize}

\begin{description}
  \item[\( \Delta\mathcal{S}=0 \)] → maximal coherence → quantum entanglement.
  \item[Increasing \( \Delta\mathcal{S} \)] → decoherence → classical behavior emerges.
  \item[Large \( \Delta\mathcal{S} \)] → coherent-gradient collapse → GR limit.
\end{description}

Thus:
\begin{align*}
&\text{Quantum Mechanics} &\leftrightarrow&\quad\text{small \( \Delta\mathcal{S} \)}\\
&\text{Classical Physics} &\leftrightarrow&\quad\text{moderate \( \Delta\mathcal{S} \)}\\
&\text{General Relativity} &\leftrightarrow&\quad\text{large \( \Delta\mathcal{S} \)}
\end{align*}

This continuous hierarchy is ESQET's unification mechanism.

\section{Golden Gravity}
We define \textbf{Golden Gravity} as curvature produced when the golden-ratio term dominates:
\[
\varphi^{-\Delta\mathcal{S}} \rightarrow \text{curvature amplification factor}.
\]
Regions with strong coherence gradients induce \emph{slightly} stronger or weaker gravity than GR predicts, depending on whether local coherence is compressing or relaxing. The effect is tiny but measurable.

\section{Four Falsifiable Predictions}

\begin{enumerate}
  \item \textbf{Entanglement-Enhanced Curvature}
  \[
  \Delta g \neq 0 \quad\text{when}\quad\mathcal{F}_{QC}\to\max
  \]
  (testable with superconducting-qubit tabletop experiments)

  \item \textbf{Scale-Dependent Gravity Drift}
  \[
  g_{\text{measured}} = g_{\text{GR}} + \epsilon(\Delta\mathcal{S})
  \]
  (micro-Cavendish experiments at cm scales)

  \item \textbf{Cosmological Coherence Polarization}
  \[
  \text{mode}_{\varphi} = \varphi^{-n}
  \]
  (detectable in upcoming CMB-S4 data)

  \item \textbf{LIGO Coherence-Mode Detection}
  Scalar coherence pulses (non-tensor) detectable by adding phase-correlation modules to LIGO’s photonic readout.
\end{enumerate}

\section{Discussion}
ESQET offers a single continuous framework spanning quantum mechanics, classical physics, relativistic gravitation, and cosmological structure formation. Its core strength is \textbf{falsifiability}: every prediction is within reach of existing or near-future instrumentation.

\section{Conclusion}
ESQET reframes the unification problem:

\begin{quote}
Spacetime is not the stage. Quantum mechanics is not the script. Both emerge from \( \mathcal{S} \), the fundamental information field.
\end{quote}

This theory—while bold—aligns with the modern turn toward quantum information as the foundation of physical law.

\begin{thebibliography}
\end{thebibliography}

\end{document}
\end{document}
